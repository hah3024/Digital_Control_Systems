\documentclass[a4paper, 12pt]{article}
\usepackage{graphicx}
\usepackage{amsmath}
\usepackage{amssymb}
\usepackage{hyperref}

\title{Inverted Pendulum Control System}
\author{Your Name}
\date{\today}

\begin{document}

\maketitle

\begin{abstract}
This report presents the design and simulation of a control system for an inverted pendulum using MATLAB. The project explores both continuous and discrete-time control strategies, including state feedback and Linear Quadratic Regulator (LQR) control. The system's behavior is analyzed under various conditions, and the effects of sampling time on stability are examined.
\end{abstract}

\tableofcontents

\section{Introduction}
The inverted pendulum is a classic problem in control theory, often used as a benchmark for testing control strategies. This project aims to stabilize the pendulum using various control techniques.

\section{System Description}
The inverted pendulum system consists of a cart with a pendulum attached. The parameters are:
\begin{itemize}
    \item Mass ($M$): 1 kg
    \item Pendulum Length ($L$): 0.842 m
    \item Friction Coefficient ($F$): 1
    \item Gravity ($g$): 9.8093 m/s²
\end{itemize}

The state vector $x$ includes:
\begin{itemize}
    \item $s$: Cart position
    \item $ds$: Cart velocity
    \item $\phi$: Pendulum angle
    \item $d\phi$: Pendulum angular velocity
\end{itemize}

\section{Control Strategies}
\subsection{State Feedback Control}
Designed using pole placement, this strategy is simulated for both linear and nonlinear systems.

\subsection{LQR Optimal Control}
The Linear Quadratic Regulator method is used to design an optimal controller, simulated for the nonlinear system.

\subsection{Discrete-Time Control}
The system is discretized, and the effects of sampling time on stability are analyzed.

\section{Simulation and Results}
\subsection{Closed-Loop System Response}
Plots of the cart position and pendulum angle over time are generated.

\subsection{Control Input}
The force applied to the cart to stabilize the pendulum is analyzed.

\subsection{Effect of Sampling Time}
The system's stability is compared for different sampling times.

\subsection{LQR Control}
Optimal control performance for the nonlinear system is demonstrated.

\section{Conclusion}
The project successfully demonstrates the stabilization of an inverted pendulum using various control strategies. The simulations provide insights into the system's behavior under different conditions.

\section{References}
\begin{itemize}
    \item MATLAB Documentation
    \item Control System Toolbox
\end{itemize}

\end{document} 